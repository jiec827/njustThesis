
\chapter{使用简介}
\label{chap:guide}

为方便使用及更好的展示\LaTeX{}排版的优秀特性,本人对模板的框架和文件体系进行了细致地处理,尽可能地对各个功能和板块进行了模块化和封装,对于初学者来说,众多的文件目录也许会让人觉得有些无所适从,但阅读完下面的使用说明后,您会发现原来使用思路是简单而清晰的,而且,当对\LaTeX{} 有一定的认识和了解后,会发现其相对Word类排版系统的极具吸引力的优秀特性。所以,如果您是初学者,请不要退缩,请稍加尝试和坚持,让自己领略到\LaTeX{}的非凡魅力。

\section{先试试效果}

\begin{equation}
a=\frac{N}{A}
\end{equation}%
\nomenclature{$a$}{The number of angels per unit area}%
\nomenclature{$N$}{The number of angels per needle point}%
\nomenclature{$A$}{The area of the needle point}%

ucasthesis模板不仅只是提供了相应的类文件,同时也提供了包括参考文献等在内的完成学位论文的一切要素,所以,下载时,推荐下载整个ucasthesis文件夹,而不是单独的文档类。

下载ucasthesis文件夹并解压后,请在文件夹内找到“Compile.bat”,双击运行,即可获得本说明文档,而这,也完成了学习使用此模板撰写论文的一半进程,什么?这就学成一半了,这么简单???,是的,就这么简单!

编译完成后,可以进入各个子目录逛逛,熟悉下模板框架。

\section{常见使用问题}

NJUSTthesis 宏包的最新版本可以从 https://latex-njust808.googlecode. com/, 网站下载。
NJUSTthesis 宏包包含两个文件:NJUSTthesis.cls 和 NJUSTthesis.cfg。简单 方便的安装方法是将宏包文件和学位论文 .tex 文件放置在同一目录下。或者将宏包 文件放置到 TEX 系统的 localtexmf/tex/latex/casthesis 目录下,然后刷新 TEX 系统 的文件名数据库。

同时,宏包还提供了一个使用模板,也就是这份说明文档的源文件。用户可以通 过修改这个模板来编写自己的学位论文。
宏包源文件 NJUSTthesisdtx.dtx完全是吴老师原来的文件,只是把原来的 CASthesis 替代 NJUSTthesis ,把 CAST 替代为 NJUST ,所以用户也可以下载吴老 师的宏包源文件 CASthesis.dtx 和 CASthesis.ins ,然后按南京理工大学博士学位 撰写要求来修订,最后再修订好的对 CASthesis.ins 文件运行 latex 编译命令来得 到宏包文件。
关于安装过程的问题可以参考 CTEX-FAQ 以及其他 LATEX 教材。 关于速成(93 分钟学会)LATEX的教材请网上下载《一份不太简短的LATEX2ε介
绍》。

为了与南京理工大学研究生院的 Word 版对照,特将南京理工大学研究生院提供 的 Word 版转化为 PDF 文件(南京理工大学学位论文撰写规范.pdf),可方便与 LATEX 版下编译出来的 PDF 进行对比。当然,在格式或尺寸方面可能有一点点差异,毕竟 两者对字体,版面等的定义存在差异,但 LATEX 版下编译出来的 PDF 完全适合《南 京理工大学学位论文撰写规范》
文件夹中的 clean.bat 文件,起到清涂编译后的临时文件用的,很方便。
此版本的参考文献采用了哈尔滨工业大学 LATEX 模版中由 jdg 编写的格式,即文 件夹中的 chinesebst2005 GBK.bst 文件。完全满足南京理工大学博士学位论文的参 考文献要求,在这里感谢 jdg 的贡献。

NJUSTthesis 宏包的最新版本可以从 https://latex-njust808.googlecode. com/, 网站下载。
NJUSTthesis 宏包包含两个文件:NJUSTthesis.cls 和 NJUSTthesis.cfg。简单 方便的安装方法是将宏包文件和学位论文 .tex 文件放置在同一目录下。或者将宏包 文件放置到 TEX 系统的 localtexmf/tex/latex/casthesis 目录下,然后刷新 TEX 系统 的文件名数据库。

同时,宏包还提供了一个使用模板,也就是这份说明文档的源文件。用户可以通 过修改这个模板来编写自己的学位论文。
宏包源文件 NJUSTthesisdtx.dtx完全是吴老师原来的文件,只是把原来的 CASthesis 替代 NJUSTthesis ,把 CAST 替代为 NJUST ,所以用户也可以下载吴老 师的宏包源文件 CASthesis.dtx 和 CASthesis.ins ,然后按南京理工大学博士学位 撰写要求来修订,最后再修订好的对 CASthesis.ins 文件运行 latex 编译命令来得 到宏包文件。
关于安装过程的问题可以参考 CTEX-FAQ 以及其他 LATEX 教材。 关于速成(93 分钟学会)LATEX的教材请网上下载《一份不太简短的LATEX2ε介
绍》。

为了与南京理工大学研究生院的 Word 版对照,特将南京理工大学研究生院提供 的 Word 版转化为 PDF 文件(南京理工大学学位论文撰写规范.pdf),可方便与 LATEX 版下编译出来的 PDF 进行对比。当然,在格式或尺寸方面可能有一点点差异,毕竟 两者对字体,版面等的定义存在差异,但 LATEX 版下编译出来的 PDF 完全适合《南 京理工大学学位论文撰写规范》
文件夹中的 clean.bat 文件,起到清涂编译后的临时文件用的,很方便。
此版本的参考文献采用了哈尔滨工业大学 LATEX 模版中由 jdg 编写的格式,即文 件夹中的 chinesebst2005 GBK.bst 文件。完全满足南京理工大学博士学位论文的参 考文献要求,在这里感谢 jdg 的贡献。

NJUSTthesis 宏包的最新版本可以从 https://latex-njust808.googlecode. com/, 网站下载。
NJUSTthesis 宏包包含两个文件:NJUSTthesis.cls 和 NJUSTthesis.cfg。简单 方便的安装方法是将宏包文件和学位论文 .tex 文件放置在同一目录下。或者将宏包 文件放置到 TEX 系统的 localtexmf/tex/latex/casthesis 目录下,然后刷新 TEX 系统 的文件名数据库。

同时,宏包还提供了一个使用模板,也就是这份说明文档的源文件。用户可以通 过修改这个模板来编写自己的学位论文。
宏包源文件 NJUSTthesisdtx.dtx完全是吴老师原来的文件,只是把原来的 CASthesis 替代 NJUSTthesis ,把 CAST 替代为 NJUST ,所以用户也可以下载吴老 师的宏包源文件 CASthesis.dtx 和 CASthesis.ins ,然后按南京理工大学博士学位 撰写要求来修订,最后再修订好的对 CASthesis.ins 文件运行 latex 编译命令来得 到宏包文件。
关于安装过程的问题可以参考 CTEX-FAQ 以及其他 LATEX 教材。 关于速成(93 分钟学会)LATEX的教材请网上下载《一份不太简短的LATEX2ε介
绍》。

为了与南京理工大学研究生院的 Word 版对照,特将南京理工大学研究生院提供 的 Word 版转化为 PDF 文件(南京理工大学学位论文撰写规范.pdf),可方便与 LATEX 版下编译出来的 PDF 进行对比。当然,在格式或尺寸方面可能有一点点差异,毕竟 两者对字体,版面等的定义存在差异,但 LATEX 版下编译出来的 PDF 完全适合《南 京理工大学学位论文撰写规范》
文件夹中的 clean.bat 文件,起到清涂编译后的临时文件用的,很方便。
此版本的参考文献采用了哈尔滨工业大学 LATEX 模版中由 jdg 编写的格式,即文 件夹中的 chinesebst2005 GBK.bst 文件。完全满足南京理工大学博士学位论文的参 考文献要求,在这里感谢 jdg 的贡献。
